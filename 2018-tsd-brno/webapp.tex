\documentclass{llncs}
%\usepackage[cp1250]{inputenc}
%\usepackage[utf8]{inputenc}
%\usepackage[czech]{babel}
\usepackage{graphicx}
%\usepackage{tipa}

\bibliographystyle{splncs}

\begin{filecontents}{citace.bib}
@inproceedings{kruuza2012making,
  title={Making Community and ASR Join Forces in Web Environment},
  author={Kr{\uu}za, Old{\v{r}}ich and Peterek, Nino},
  booktitle={International Conference on Text, Speech and Dialogue},
  pages={415--421},
  year={2012},
  organization={Springer}
}
\end{filecontents}



\begin{document}
\newtheorem{Definition}{Definition}
\title{ASR-Output-Correcting Webapp After 6 Years}

\author{Oldřich Krůza and Vladislav Kuboň}
\institute{Charles University in Prague\\
           Faculty of Mathematics and Physics\\
           Institute of Formal and Applied Linguistics\\
           Malostranské nám. 25, Prague, Czech Republic\\
           \{kruza,vk\}@ufal.mff.cuni.cz}

\maketitle

\begin{abstract}

This paper presents a next-generation web application that enables users to
contribute corrections to automatically acquired transcription of long speech
recordings. We describe differences from similar settings, compare our solution
with others and reflect on the development from the first, 6 years old version
in the light of the work done, lessons learned and the new technologies
available in the browser.

\end{abstract}

\section{Introduction}

In 2012\cite{kruuza2012making}, we presented a setting where a community of
users contributed corrections to automatically transcribed talks of a single
speaker. Now that the browser technologies evolved drastically and we could
observe the usage patterns and discover shortcomings of the solution at hand, we
have created a next generation of the programme. We shall describe the steps
taken and discuss their motivation and impact.

\section{}

\section{Conclusion}

\section*{Acknowledgments}

The research was supported by SVV project number 265 314.\\
\\
This work has been using language resources stored
by the LINDAT-Clarin project of the Ministry of
Education of the Czech Republic (project LM2010013).
%\bibliography{paper}

\bibliography{citace}

\end{document}
