\documentclass{llncs}
%\usepackage[cp1250]{inputenc}
%\usepackage[utf8]{inputenc}
%\usepackage[czech]{babel}
\usepackage{graphicx}
%\usepackage{tipa}

\bibliographystyle{splncs}

\begin{filecontents}{citace.bib}
@inproceedings{kruuza2012making,
  title={Making Community and ASR Join Forces in Web Environment},
  author={Kr{\uu}za, Old{\v{r}}ich and Peterek, Nino},
  booktitle={International Conference on Text, Speech and Dialogue},
  pages={415--421},
  year={2012},
  organization={Springer}
}
\end{filecontents}



\begin{document}
\newtheorem{Definition}{Definition}
\title{ASR-Output-Correcting Webapp After 6 Years}

\author{Oldřich Krůza and Vladislav Kuboň}
\institute{Charles University in Prague\\
           Faculty of Mathematics and Physics\\
           Institute of Formal and Applied Linguistics\\
           Malostranské nám. 25, Prague, Czech Republic\\
           \{kruza,vk\}@ufal.mff.cuni.cz}

\maketitle

\begin{abstract}

This paper presents a next-generation web application that enables users to
contribute corrections to automatically acquired transcription of long speech
recordings. We describe differences from similar settings, compare our solution
with others and reflect on the development from the first, 6 years old version
in the light of the work done, lessons learned and the new technologies
available in the browser.

\end{abstract}

\section{Introduction}

In 2012\cite{kruuza2012making}, we presented a setting where a community of
users contributed corrections to automatically transcribed talks of a single
speaker. Now that the browser technologies evolved drastically and we could
observe the usage patterns and discover shortcomings of the solution at hand, we
have created a next generation of the programme. We shall describe the steps
taken and discuss their motivation and impact.

\section{Differences to other settings}

In our setting, we have a large spoken corpus (about 1000 hours) of a single
speaker. Our aim is to have a transcription as good as possible for the purpose
of searching and further, higher-level processing of the data. There is a pool
of people interested in the talks, who on one hand are the force we can try to
employ and on the other hand are the consumers of our effort, our target group
so to speak.

The web application should therefore combine the two purposes: 1. serve its user
with making the content available in a manner as good as possible and 2. animate
the user to give as much and as high-quality contribution as possible.

To our best knowledge, there is no other project with a comparable setting.
However, we can compare single aspects found in other applications.

\subsection{Transcribing apps}

The main differences to common transcribing software are that
\begin{enumerate}
\item{they are optimized for the case where there is no transcription available
and it must be acquired from scratch;}
\item{no quality control: the user is free to enter whatever transcription she
pleases and the ultimate measure is her content; in contrast, we need the
transcription to be accurate because it is used as training data for the
acoustic model;}
\item{the alignment is on the level of phrases, if any; we use alignment on the
level of phones;}
\item{standalone transcribing applications are user-centric: the user
transcribes whatever acoustic data they choose; our system is data-centric: the
whole application with all its tools and persons revolves around the data set;}
\item{common transcription software assumes the user wants to transcribe; we
assume the user want to listen and possibly read along and we want to animate
her to submit transcriptions.}
\item{user collision}
\end{enumerate}

Our first reference is
Transcriber\footnote{sourceforge.net/projects/trans}. Transcriber is an
open-source program written in TCL.

\section{Conclusion}

\section*{Acknowledgments}

The research was supported by SVV project number 265 314.\\
\\
This work has been using language resources stored
by the LINDAT-Clarin project of the Ministry of
Education of the Czech Republic (project LM2010013).
%\bibliography{paper}

\bibliography{citace}

\end{document}
