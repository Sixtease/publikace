\begin{document}
\begin{abstract}
\end{abstract}

\section{Introduction}

The way a volume of data is stored impacts all subsequential processing and
exploitation. A decision on storing the data has to be made as soon as the data
is being acquired. Obviously, the difficulty of changing the way of storing the
data increases with the volume of the data stored and with parties relying on
it. The relying parties can be in-house software tools or even public
interfaces\footnote{I hesitate to say API because we're talking about access to
data, not to applications}.

In our case, the data at hand is a collection of speech recordings of total
magnitude of about one thousand hours. The decisions to be made span many
levels and the highest levels dictate further questions. Will the data be stored
as files on a traditional file system or in a database or in a yet another way?
Will the whole data set be contained on one physical storage device or will it
be distributed over several?

I have chosen the traditional way of using files on a single directory tree, so
the next questions are: How will the audio material be divided into individual
files, what will be the directory structure, what will be the file naming and
optionally what additional metadata will be maintained.

In the following chapters, I shall list the original structuring for the Spoken
corpus of Karel Makoň, provide reasons that led to that structuring, explain why
a change was necessary, and describe the journey to realizing it.

\section{Original Structuring}

\subsection{Prior to Digitization}

The corpus arose as a set of amateur recordings on magnetic tapes. The author of
the talks, Mr. Karel Makoň, a Czech mystic, started giving talks in private
groups after his deportation into a concentration camp in 1939, and stopped in
1991, two years before his death. His regular listeners were recording his
talks. I don't know exactly when the recording started but the earliest denoted
date is in 1973.

A part of the recordings (30\% of total length) was taken onto reels, the rest
onto cassettes. Mostly, there was an identifier for a reel tape or for a
cassette. Some pieces were unlabeled and some shared a label.

The majority of the recordings were cassettes taken by a single person with
identifiers of the format YY-NN where YY are the two digits of the year and NN
is the ordinal number in that year, so for example {\texttt 85-05} is the fifth
recording taken in the year 1985. These occupy 686 resulting files out of
total 802 files originating in cassettes.

\subsection{Digitization}

Digitization of the tapes is naturally done one side of the tape into one file.
Any merging or splitting is extra work. Thus, most of the files resulting from
the digitization process correspond to one side of a magnetophone cassette
(worth mostly 45 minutes) or one pass from reel to reel in 9cm/s (90min).

During the two years of digitization, some experiments aiming at saving time and
effort were done. For instance, using a device that can auto-reverse and play
two cassettes in a row.

\section{Need for Change}

\section{New Structuring}

\subsection{Where It Is and Isn't Used}

\subsection{}

\end{document}
